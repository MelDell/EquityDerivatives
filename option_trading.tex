

\documentclass[12pt]{article}


\usepackage{amsmath}
\usepackage{nccmath}
\usepackage{mathtools }
\usepackage{amsthm}
\usepackage{hyperref}
\usepackage{xcolor}

\everymath{\displaystyle}

\newcommand{\der}[2]{\frac{\mathrm{d}{#1}}{\mathrm{d}{#2}}}


\begin{document}
\section{Black and Scholes Call Option Pricing}
We assume rates are zero and dividends are zero. We also assume that the stock price follows a geometric Brownian motion. The price of the call option at time $t$ is given by the Black and Scholes formula
\begin{fleqn}
\begin{equation}
\begin{aligned}
\text{Let } S_t &\text{ be the price of the stock at time } t,\\
K &\text{ be the strike price of the option,}\\
\sigma &\text{ be the volatility of the stock,}\\
T &\text{ be the time to maturity of the option,}\\
C_t &\text{ be the price of the call option at time } t.
\end{aligned}
\end{equation}
\end{fleqn}
the price of the call option at time $t$ is given by the Black and Scholes formula
\[\boxed{C_t = S_tN(d_1) - KN(d_2)}\]
where $N(x)$ is the cumulative distribution function of the standard normal distribution
\subsection{Greeks}
\begin{itemize}
\item Delta ; $\Delta = \der{C}{S}=N(d_1)$
\item Gamma ; $\Gamma = \der{^2C}{S^2}=\frac{N'(d_1)}{S\sigma\sqrt{T}}$
\end{itemize}
Applying Taylor expansion to the Black and Scholes formula, we get
\begin{equation}
\begin{aligned}
C_t(S_2) -C_t(S_1) &= \Delta_{S_1}(S_2-S_1) + \frac{1}{2}\Gamma_{S_1}(S_2-S_1)^2 + \cdots
\end{aligned}
\end{equation}
if t has changed we need to add the theta
\begin{equation}
\begin{aligned}
C_{t_2}(S_2) -C_{t_1}(S_1) &= + \theta_{S_1,t_1}(t_2-t_1) +\Delta_{S_1}(S_2-S_1) + \frac{1}{2}\Gamma_{S_1}(S_2-S_1)^2 + \cdots
\end{aligned}
\end{equation}

\newpage
\subsection{Example}
Assume that $S_1=100$, $K=100$ $T=30$, and $v=20\%$
\\the price of the call option at $S_1$ is given by
\[C_1 = 100N(d_1) - 100N(d_2)=2.7524\]
\\s=100, k=100, t=30, v=0.2
\\Call Price: 2.7524, delta=0.5138,gamma=0.0578, theta=11.5556, vega=13.76
\\ Assume you bought 100 calls.
\begin{itemize}
    \item Premium spent = $100\times 2.7524 = 275.24$
    \item Delta = $100\times 0.5138 = 51.38$ shares
    \item Gamma = $100\times 0.0578 = 5.78$
\end{itemize}
\subsubsection*{Small change in stock price}
Now assume that the stock price increases to $S_2=101$
\\s=101, k=100, t=30, v=0.2
\\Call Price: 3.2949, delta=0.5709,gamma=0.0563, theta=11.4931, vega=13.68
\begin{itemize}
    \item Premium = $100\times 3.2949 = 329.49$
    \item Delta = $100\times 0.5709 = 57.09$ shares
    \item Gamma = $100\times 0.0563 = 5.63$
\end{itemize}
so if you bougt the 100 calls at 2.7524, you spent \$275.24.
\\If the stock price increases to 101, the value of the calls increases to 3.2949. so you can sell your 100 calls for \$329.49
\\The profit is \$329.49 - \$275.24 = \$54.25
\\using Taylor expansion, we can approximate the profit as
\begin{equation}
\begin{aligned}
C_2 - C_1 &= \Delta_{S_1}(S_2-S_1) + \frac{1}{2}\Gamma_{S_1}(S_2-S_1)^2\\
&= 51.38(101-100) + \frac{1}{2}\times 5.78(101-100)^2\\
&= 51.38 + 2.89 = 54.27
\end{aligned}
\end{equation}
if you have delta hedged the calls, you sold 51.38 shares of the stock at 100. Now those shares are worth 101, so you lost
\[ 51.38\times(101-100) = 51.38\]
your total position is
\begin{itemize}
    \item 100 calls bougt at 2.7524, now worth 3.2949, hence a profit of 54.25
    \item 51.38 shares sold at 100, now worth 101, hence a loss of 51.38
    \item total profit = 54.25 - 51.38 = 2.87
\end{itemize}

The profit from the hedged position is :
\[54.25 - 51.38 = 2.87\]
which is very close to the expected profit from the gamma
\[ \frac{1}{2}\Gamma_{S_1}(S_2-S_1)^2 =\frac{1}{2}\times 5.78\times(101-100)^2 = 2.89\]
\\ It also worth noticing that
\begin{itemize}
\item the initial delta was 51.38 shares,
\item the final delta was 57.09 shares.
\item delta change = final delta - initial delta = 57.09 - 51.38 = 5.71
\item final delta-initial delta :$\Delta_{S_2}-\Delta_{S_1}\sim \Gamma_{S_1}\times(S_2-S_1)$ 
\item $\Gamma_{S_1}\times(S_2-S_1)=5.78\times(101-100)=5.78 $
\end{itemize}
Since the delta keeps changing, the delta hedging can not be perfect. The delta hedging is only accurate for small changes in the stock price.
\subsubsection*{Time Decay}
However, the price of the option decreases as time passes. The rate of decrease is given by the theta
\[\theta = \der{C}{t} = -\frac{S_tN'(d_1)\sigma}{2\sqrt{T}}\]
In this formula T is in years, so if t=30, then T=30/252.
Here the theta is 11.5556, so the value of each option decreases by $11.56\times\frac{1}{252}=0.0459$ per day.
\\since you have 100 options, the value of your options decreases by $0.0459\times 100 = 4.59$ per day.
so your net profit is $2.87 - 4.59 = -1.72$
\\The stock price did not change enough to cover the time decay.
\\Remeber that our implied volatility is 20\%, so we expect daily moves of $\frac{0.2}{\sqrt{252}}\times 100 = 1.26$. So we will only make money if the spots moves more than \$1.26 
\subsubsection*{Large change in stock price}
suppose now that the stock price increases to 110
\\s=110, k=100, t=30, v=0.2
\\Call Price: 10.2765, delta=0.9216,gamma=0.0193, theta=4.69, vega=5.56
\begin{itemize}
    \item Premium = $100\times 10.2765 = 1027.65$
    \item Delta = $100\times 0.9216 = 92.16$ shares
    \item Gamma = $100\times 0.0193 = 1.93$
\end{itemize}
If the stock price increases to 110, the value of the calls increases to 10.2765. so you can sell the calls for \$1,027.65
\\The profit is \$1,027.65 - \$275.24 = \$752.41
if you have delta hedged the calls, you sold 51.38 shares of the stock at 100. Now those shares are worth 110, so you lost
\[ 51.38\times(110-100) = 513.8\]
your total position is
\begin{itemize}
    \item 100 calls bougt at 2.7524, now worth 10.2765, hence a profit of 752.41
    \item 51.38 shares sold at 100, now worth 110, hence a loss of 513.8
    \item total profit = 752.41 - 513.8 = 238.61
\end{itemize}

The profit from the calls is 752.41, so the total profit is
\[752.41 - 513.8 = 238.61\]

The time decay is still 4.69$0.0459\times 100 = 4.59$ per day, so the total profit is
\[238.61 - 4.69 = 233.92\]
The stock price increased enough to cover the time decay.
\\again the profit can be approximated by the gamma
\begin{equation}
\begin{aligned}
C_2 - C_1 &= \theta +\frac{1}{2}\Gamma(S_1)(S_2-S_1)^2\\
&= -4.59+ \frac{1}{2}\times 5.78(110-100)^2\\ 
&= -4.59+ \frac12 \times 578 = 284.41
\end{aligned}
\end{equation}
Here the Taylor expansion is not as accurate because the stock price increased by 10, which is a large change. The gamma is only an approximation, and it is only accurate for small changes in the stock price.
\\remember that our break-even point is 1.26, so we will only make money if the stock price moves more than 1.26. Here the move is 10/1.26=7.93, so we made money.
\\More specifically, our gamma PnL is roughly $(7.93)^2\times \theta=62.88\times 4.59=288$ 
\section*{Useful facts}
\begin{itemize}
\item the price of the call can never be greater than the stock price
\item the price of the put can never be greater than the strike price
\item a good approximation for the delta is the probability that the option will be in the money at expiration
\item a good approximation for the atm call is
\[C=0.4S\sigma\sqrt{T} \text{ with T in years}\]
\item in particular if $T_2=4\times T_1$ then $C_2=2\times C_1$
\item a good approximation for the vega of the atm call is
\[C=0.4S\sqrt{T} \text{ with T in years}\]
\item So both the price and the vega of the atm call are proportional to the square root of the time to maturity
\item Call-Put parity: $C-P=S-K$
\end{itemize}

\subsection*{Put-Call Parity}
suppose you have a call option with a strike price of K and a put option with a strike price of K. The call option is worth $C$ and the put option is worth $P$. The stock price is $S$.
suppose you buy the call and sell the put. The payoff is
\begin{itemize}
\item if the stock price is greater than K, the call is worth $S-K$ and the put is worth 0, so the payoff is $S-K$
\item if the stock price is less than K, the call is worth 0 and the put is worth $K-S$, so the payoff is $S-K$
\item so the payoff is always $S-K$ not matter what the stock price is
\item hence
\[C-P=S-K\]
\item the call and the put have different deltas but the same gamma, same theta, and same vega
\end{itemize}


\end{document}